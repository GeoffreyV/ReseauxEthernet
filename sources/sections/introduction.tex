\section{Qu'est-ce qu'un réseau ? Qu'est-ce qu'internet ?}
\subsection{Un réseau}

De tout temps, les humains ont ressenti et ressentent le besoin de communiquer.
Les méthodes dont nous nous servons pour partager idées et informations changent et évoluent sans cesse. Si le réseau humain se limitait autrefois à des conversations en face à face, aujourd'hui les découvertes en matière de supports étendent sans cesse la portée de nos communications. De la presse écrite à la télévision, chaque innovation a développé et amélioré nos moyens de communication.

Avant de commencer ce cours sur les réseaux informatiques, nous devons définir ce qu'est un réseau.
% A retenir
\UPSTIdefinition[Réseau]{\UPSTIlignesACompleter{ Un réseau désigne un ensemble de relations. Définir un réseau, c'est donc définir des \textbf{liens}. D'une façon plus concrète, on dira : \textbf{Un réseau est un groupe d'entités en communication}}}

Une \textit{entité} peut être de tout type. Dans le cas d'un réseau social, les entités sont les personnes appartenant à un groupe social. Dans le cas d'un réseau informatique, les entités sont les appareil informatiques connectés ensembles.

En résumé, parler de réseau c'est simplement parler d'entités qui communiquent. La problématique de ce cours pourrait donc s'écrire :

\textbf{Comment faire en sorte que différents composants informatiques s'échangent des informations ? }

\subsection{Internet}

Ethymologiquement, Internet vient de \textit{inter} (entre) et \textit{net} (réseau). Il s'agit donc d'une connexion de multiples réseaux entre eux.

\UPSTIdefinition[Internet]{
Internet est un ensemble de réseaux interconnectés à l'échelle internationale qui échangent des informations selon des normes communes en utilisant des câbles téléphoniques, des câbles à fibre optique, des transmissions sans fil et des liaisons par satellite.
}



\subsection{Les protocoles}
Pour pouvoir donner une information à quelqu'un, il faut être capable de le joindre mais également de parler la même langue que lui (ou en tout cas que les différents intermédiaires comprennent leurs langues respectives).

En informatique, on ne parle pas de langue mais de protocole. Les machines communiquent en respectant un même protocole.

\UPSTIaRetenir{Les protocoles de communication définissent les "rêgles" que doivent respecter les machines pour communiquer.}
