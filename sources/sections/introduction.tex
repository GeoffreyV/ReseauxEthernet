\section{Généralités sur les réseaux}
\subsection{Définitions}

De tout temps, les humains ont ressenti et ressentent le besoin de communiquer.
Les méthodes dont nous nous servons pour partager idées et informations changent et évoluent sans cesse. Si le réseau humain se limitait autrefois à des conversations en face à face, aujourd'hui les découvertes en matière de supports étendent sans cesse la portée de nos communications. De la presse écrite à la télévision, chaque innovation a développé et amélioré nos moyens de communication.

Avant de commencer ce cours sur les réseaux informatiques, nous devons définir ce qu'est un réseau.

\UPSTIdefinition[Réseau]{\UPSTIlignesACompleter{ Un réseau désigne un ensemble de relations. Définir un réseau, c'est donc définir des \textbf{liens}. D'une façon plus concrète, on dira : \textbf{Un réseau est un groupe d'entités en communication}}}

Une \textit{entité} peut être de tout type. Dans le cas d'un réseau social, les entités sont les personnes appartenant à un groupe social. Dans le cas d'un réseau informatique, les entités sont les appareil informatiques connectés ensembles.

En résumé, parler de réseau c'est simplement parler d'entités qui communiquent. Lorsque des entités sont des personnes vivant en société on parlera de \textit{réseau social} (\UPSTIfigure{\ref{fig:reseauSocial}}). Lorsque ce sont des appareils informatiques, on parlera de réseaux informatiques \UPSTIfigure{\ref{fig:reseauInformatique}} La problématique de ce cours pourrait donc s'écrire : \textbf{Comment faire en sorte que différents composants informatiques s'échangent des informations ? }

\begin{figure}[ht]
  \begin{subfigure}{.45\textwidth}
    \caption{Un réseau social}
    \label{fig:reseauSocial}
  \end{subfigure}
  \begin{subfigure}{.45\textwidth}
    \caption{Un réseau informatique}
    \label{fig:reseauInformatique}
  \end{subfigure}
  \caption{Réseau social et réseau informatique}
\end{figure}

En réseau informatique, chaque élément d'un réseau est appelé un  \hyperlink{DefNoeud}{noeud}.

\UPSTIdefinition[Noeud]{\hypertarget{DefNoeud}{Un noeud est un ordinateur ou toute entité connectée à un réseau par l'intermédiaire d'une carte réseau.}}


\subsection{Caractéristiques d'un réseau}

\subsubsection{Débit et bande passante}
Lorsque l'on traite de l'échange d'informations entre deux entités, on caractérise la quantité de données que cette connexion transmet par unité de temps. Cette quantité est caractérisée par deux grandeur que sont la \textbf{bande passante} et le \textbf{débit}.

\UPSTIdefinition[Temps bit]{Le temps bit est la durée d'un bit dans le signal transmettant une donnée.}

\UPSTIdefinition[Débit]{Le débit d'une transmission est la quantité d'information \textbf{réellement transmise} par unité de temps. Elle s'exprime en \si{bit/s} ou \si{octet/s}. On calcule le débit $D$ à partir du temps bit : $D=\frac{1}{t_{\text{bit}}}$. }

\UPSTIdefinition[Bande passante]{La bande passante caractérise la laisone en elle-même entre deux organes. Elle est la quantité d'informations \textbf{maximale} transmissible sur une connexion. }

\subsubsection{Latence}
\UPSTIdefinition[latence]{En informatique, la latence (ou délai de transit, ou retard) est le délai de transmission dans les communications informatiques (on trouve parfois l’anglicisme lag).

Il désigne le temps nécessaire à un paquet de données pour passer de la source à la destination à travers un réseau. }

Le débit et la bande passante sont à ne pas confondre avec la latence. Il est possible d'avoir un débit très élevé dans un réseau avec une forte latence. Cela signifierait qu'une grande quantité d'information est transmise chaque seconde, mais que ces informations mettent beaucoup de temps à réaliser le trajet entre les deux noeuds.

\subsubsection{Caractériser un réseau}
On peut définir différentes caractéristiques d'un réseau. Selon les applications, ces caractéristiques seront plus ou moins importantes.
\begin{description}
  \item [Fiabilité : ] capacité d'un réseau à acheminer les messages sans erreur
  \item [Robustesse : ] capacité d'un réseau à ne pas tomber en panne
  \item [Rapidité : ] la rapidité est d'autant plus forte que la latence est faible.
  \item [Déterminisme : ] connaissance précise du temps de réponse
  \item [Débit : ] Quantité de données pouvant être acheminée chaque seconde
\end{description}
