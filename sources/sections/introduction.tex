\section{Généralités sur les réseaux}
\subsection{Définitions}

De tout temps, les humains ont ressenti et ressentent le besoin de communiquer.
Les méthodes dont nous nous servons pour partager idées et informations changent et évoluent sans cesse. Si le réseau humain se limitait autrefois à des conversations en face à face, aujourd'hui les découvertes en matière de supports étendent sans cesse la portée de nos communications. De la presse écrite à la télévision, chaque innovation a développé et amélioré nos moyens de communication.

Avant de commencer ce cours sur les réseaux informatiques, nous devons définir ce qu'est un réseau.
% A retenir
\UPSTIdefinition[Réseau]{\UPSTIlignesACompleter{ Un réseau désigne un ensemble de relations. Définir un réseau, c'est donc définir des \textbf{liens}. D'une façon plus concrète, on dira : \textbf{Un réseau est un groupe d'entités en communication}}}

Une \textit{entité} peut être de tout type. Dans le cas d'un réseau social, les entités sont les personnes appartenant à un groupe social. Dans le cas d'un réseau informatique, les entités sont les appareil informatiques connectés ensembles.

En résumé, parler de réseau c'est simplement parler d'entités qui communiquent. La problématique de ce cours pourrait donc s'écrire :

\textbf{Comment faire en sorte que différents composants informatiques s'échangent des informations ? }


\UPSTIdefinition[Noeud]{Un noeud est un ordinateur ou toute entité connectée à un réseau par l'intermédiaire d'une carte réseau.}

\subsection{Caractéristiques d'un réseau}
On peut définir différentes caractéristiques d'un réseau. Selon les applications, ces caractéristiques seront plus ou moins importantes.
\begin{description}
  \item [Fiabilité : ] capacité d'un réseau à acheminer les messages sans erreur
  \item [Robustesse : ] capacité d'un réseau à ne pas tomber en panne
  \item [Déterminisme : ] connaissance précise du temps de réponse
  \item [Rapidité : ] temps de réponse
  \item [Débit : ] Quantité de données pouvant être acheminée chaque seconde
\end{description}

\UPSTIdefinition{\begin{itemize}
  \item Le \textbf{temps-bit} est la durée de transmission d'un bit
  \item Le \textbf{débit} est la quantité d'information transmise par unité de temps.
  \begin{itemize}
    \item Exprimé en bit/s ou octet/s
  \end{itemize}
\end{itemize}}
