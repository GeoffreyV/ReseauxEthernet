
\subsection{Internet}

Ethymologiquement, Internet vient de \textit{inter} (entre) et \textit{net} (réseau). Il s'agit donc d'une connexion de multiples réseaux entre eux.




\UPSTIdefinition[Internet]{\hypertarget{DefInternet}{
Internet est un ensemble de réseaux interconnectés à l'échelle internationale qui échangent des informations selon des normes communes en utilisant des câbles téléphoniques, des câbles à fibre optique, des transmissions sans fil et des liaisons par satellite.
}}

Chacun de ces réseaux a sa propre topologie (la plus répandue étant une topologie en étoile). Cette diversité de topologies et le fait que plusieurs chemins soient possibles entre deux noeuds font qu'internet a une topologie hybride.


\subsection{Les protocoles}
Pour pouvoir donner une information à quelqu'un, il faut être capable de le joindre mais également de parler la même langue que lui (ou en tout cas que les différents intermédiaires comprennent leurs langues respectives).

En informatique, on ne parle pas de langue mais de protocole. Les machines communiquent en respectant un même protocole.

\UPSTIaRetenir{Les protocoles de communication définissent les "rêgles" que doivent respecter les machines pour communiquer.}
