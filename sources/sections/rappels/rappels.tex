\section*{Quelques rappels}
\UPSTIdefinition[Donnée]{En informatique, une donnée est la représentation \textit{numérique} d'une information. Une donnée est une valeur représentant quelque chose.}

\UPSTIrappel{\begin{description}
  \item [bit] Un bit est un symbôle pouvant prendre deux valeurs : \textbf{1} ou un \textbf{0}.
  \item [octet] Un octet est un ensemble de 8 bits. On peut le représenter sous la forme de deux caractères hexadécimaux : FF = 1111 1111.
  \item [byte] Le mot anglais \textbf{byte} se traduit par \textit{multiplet}. Il désigne la plus petite unité « logiquement » adressable par un programme. De nos jours, un \textbf{byte} désigne presque toujours un octet. Attention à ne pas le confondre avec un bit. $\SI{1}{byte} = \SI{8}{bits}$
\end{description}}

\begin{table}[h!t]
  \centering
  \begin{tabular}{c|c|c}
          &   bit       & octet \\ \hline
  \SI{1}{kb} &  \SI{1 000}{bits}              &  $\frac{1000}{8} = \SI{125}{octets} $\\
  \SI{1}{Mb} &  \SI{1 000 000}{bits}          &  $\frac{1000000}{8} = \SI{125 000}{octets} $\\
  \SI{1}{Gb} &   \SI{1 000 000 000}{bits}     &  $\frac{1000000000}{8} = \SI{125 000 000}{octets} $\\
  \SI{1}{Tb} &  \SI{1 000 000 000 000}{bits}  &  $\frac{1000000000000}{8} = \SI{125 000 000 000}{octets} $\\
  \hline
  $\SI{1}{ko} =  \SI{1}{kB}$  &  \SI{8 000}{bits}              &  \SI{1 000}{octets}\\
  $\SI{1}{Mo} =  \SI{1}{MB}$  &  \SI{8 000 000}{bits}          &  \SI{1 000 000}{octets}\\
  $\SI{1}{Go} =  \SI{1}{GB}$  &   \SI{8 000 000 000}{bits}     &  \SI{1 000 000 000}{octets}\\
  $\SI{1}{To} =  \SI{1}{TB}$   &  \SI{8 000 000 000 000}{bits}  &  \SI{1 000 000 000 000}{octets}\\
\end{tabular}

  \caption{Tableau de conversion quantité de données}
\end{table}
