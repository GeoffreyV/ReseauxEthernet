\subsection{Clients et serveurs}

Tous les ordinateurs connectés à un réseau et qui participent directement aux communications transmises sur le réseau sont des hôtes. Les hôtes peuvent envoyer et recevoir des messages sur le réseau. Dans les réseaux actuels, les ordinateurs hôtes peuvent jouer le rôle de client, de serveur, ou les deux. Les logiciels installés sur l'ordinateur déterminent le rôle qu'il tient au sein du réseau.

\paragraph{Les serveurs}
\UPSTIdefinition[Serveurs]{Les serveurs sont des hôtes équipés des logiciels leur permettant de fournir des informations, comme des messages électroniques ou des pages Web, à d'autres hôtes sur le réseau.} Chaque service nécessite un logiciel serveur distinct. Par exemple, un hôte nécessite un logiciel de serveur Web pour pouvoir offrir des services Web au réseau. Chaque destination que vous visitez en ligne vous est fournie par un serveur situé quelque part sur un réseau qui est connecté à Internet.

\paragraph{Les clients}
\UPSTIdefinition[Clients]{Les clients sont des ordinateurs hôtes équipés d'un logiciel qui leur permet de demander des informations auprès du serveur et de les utiliser.}

Un navigateur est un exemple de logiciel client, comme Internet Explorer, Safari, Mozilla Firefox ou Chrome.

\begin{UPSTIactivite}Associer les clients aux serveurs correspondant dans la section \href{https://static-course-assets.s3.amazonaws.com/NetEss/fr/index.html#1.2.1.3}{1.2.1.3} du cours netacad.

\end{UPSTIactivite}
