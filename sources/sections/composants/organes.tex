\subsection{Périphériques finaux (ou terminaux)}
\label{sec:organes}
\UPSTIdefinition[Périphérique final]{Les périphériques finaux (ou terminaux) sont les périphériques qui envoient ou reçoivent des trames sur un réseau. Ce sont les organes qui \textbf{utilisent} le réseau pour communiquer. On parle aussi d'hôte du réseau.}

\subsubsection{La carte réseau}
\label{sec:CarteReseau}
Un ordinateur (ou n'importe quel appareil informatique) communique via sa carte réseau. C'est elle qui fait l'interface entre l'ordinateur et le réseau en lui-même.

Lorsque votre ordinateur communique sur le réseau, c'est en réalité sa carte réseau qui le fait. De même, les paquets (messages) sont adressés à cette carte réseau.

Chaque carte réseau possède un identifiant \textbf{unique} appelé \textbf{adresse MAC}. C'est sa carte d'identité, ce qui l'identifie au sein d'un réseau. Nous en reparlerons quand nous parlerons du protocole de communication Ethernet.

\begin{figure}[h]
\centering
  \includegraphics[width=.4\textwidth]{images/materiel/carteReseau}
  \caption{Carte réseau}
  \label{fig:carteReseau}
\end{figure}

\subsection{Périphériques intermédiaires}
\UPSTIdefinition[Periphérique intermédiaire]{Ce sont les appareils qui \textbf{permettent l'acheminement} des messages sans en être utilisateurs.}
\subsubsection{Concentrateur (hub)}
\label{sec:concentrateur}
Un concentrateur est un dispositif qui permet de relier différents organes (ordinateurs) entre-eux. Chaque organe est relié sur un port (prise RJ45) différent. On peut le voir comme une simple "multi-prise" qui se contente d'envoyer tout ce qu'elle reçoit sur un port sur les autres ports.

Par exemple, si quatre ordinateurs sont connectés sur les ports 1, 2, 3 et 4 et que l'ordinateur branché sur le port 4 envoie un paquet à celui connecté sur le port 2, le hub le transmettra sur les ports 1, 2 et 3 sans se soucier du destinataire du message.

\subsubsection{Commutateur (switch)}
\label{sec:commutateur}
Un commutateur sert également à relier différents organes entre-eux au sein d'un réseau. Cependant, contrairement au concentrateur, le commutateur ne recopie pas tous les messages qu'il reçoit sur tous les ports. En effet, il n'enverra le paquet uniquement au port sur lequel est connecté son destinataire.

En reprenant l'exemple précédent, lorsque l'ordinateur connecté au port 4 désire envoyer un paquet à celui connecté au port 2, le commutateur transmettra ce message sur le port 2 et non sur les autres ports.

Pour identifier les éléments connectés sur ses ports, le commutateur utilise l'adresse MAC (l'identifiant unique attribué aux cartes réseaux).

\subsubsection{Routeur}
\label{sec:routeur}
Un routeur, quant à lui, permet non seulement de connecter des organes entre-eux, mais il fait également le lien entre deux réseaux différents. L'exemple le plus parlant est sûrement la box internet que chacun à chez soi. Elle permet de relier les différents ordinateur du domicile entre-eux et permet également à ces ordinateur d'accéder à un réseau bien plus étendu : internet.
