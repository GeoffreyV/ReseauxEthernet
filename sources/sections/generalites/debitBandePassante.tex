Lorsque l'on traite de l'échange d'informations entre deux entités, on caractérise la quantité de données que cette connexion transmet par unité de temps. Cette quantité est caractérisée par deux grandeur que sont la \textbf{bande passante} et le \textbf{débit}.

\UPSTIdefinition[Temps bit]{Le temps bit est la durée d'un bit dans le signal transmettant une donnée.}

\UPSTIdefinition[Débit]{Le débit d'une transmission est la quantité d'information \textbf{réellement transmise} par unité de temps. Elle s'exprime en \si{bit/s} ou \si{octet/s}. On calcule le débit $D$ à partir du temps bit : $D=\frac{1}{t_{\text{bit}}}$. }

\UPSTIdefinition[Bande passante]{La bande passante caractérise la laisone en elle-même entre deux organes. Elle est la quantité d'informations \textbf{maximale} transmissible sur une connexion. }
